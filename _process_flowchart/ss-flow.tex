% building on poster template, Ty Stanford 2015-08-21

% \documentclass[a2,landscape]{a0poster}
\documentclass[a2,portrait]{a0poster}
%a2,a1,a0 are all possible

\usepackage[absolute]{textpos}
\usepackage{graphicx}
\usepackage{xcolor}
\usepackage{tikz}
\usepackage{fancybox}
\usepackage{fancyvrb}
\usepackage[utf8]{inputenc}
\usepackage[scaled=.8]{beramono}
\usepackage[T1]{fontenc}
\usepackage{array}
\usepackage{dsfont}
\usepackage{amsmath,enumerate,amssymb,amsthm}
% \usepackage{bm}
% \usepackage{varioref}
% \usepackage[sort&compress]{natbib}
\usepackage{enumitem}
% \usepackage{pagecolor}% http://ctan.org/pkg/pagecolor


%%%%%%%%%%%%%%%%%%%%%%%%%%% packages options %%%%%%%%%%%%%%%%%%%%%%%%%%%%%%%

\setenumerate[1]{label=(\arabic*)}
% \bibpunct{[}{]}{;}{s}{,}{,}
% \def\bibfont{\footnotesize}
\usetikzlibrary{arrows,arrows.meta}


%%%%%%%%%%%%%%%%%%%%%%%%%%% general settings %%%%%%%%%%%%%%%%%%%%%%%%%%%%%%%


\pagestyle{empty}
\renewcommand*{\familydefault}{\sfdefault}

\parindent=0pt
\parskip=0.5\baselineskip


%%%%%%%%%%%%%%%%%%%%%%%%%%% grid for poster %%%%%%%%%%%%%%%%%%%%%%%%%%%%%%%

\TPGrid[10mm,10mm]{4}{100}      % 4 cols and 100 rows, 
% \TPGrid[10mm,10mm]{5}{100}      % 5 cols and 100 rows, 
\TPMargin{10mm}                % 10mm inside each grid the boxes are formed
\setlength{\TPboxrulesize}{4pt}



%%%%%%%%%%%%%%%%%%%%%%%%%%% colour defns (add as req) %%%%%%%%%%%%%%%%%%%%%%%%%%%%%%%


\definecolor{pinkishoffwhite}{rgb}{0.950,0.869,0.753}
\definecolor{redmaroon}{rgb}{0.451,0.086,0.088}
\definecolor{uniblue}{rgb}{0.157,0.322,0.614}
\definecolor{midgrey}{rgb}{0.220,0.220,0.251}

\definecolor{tgreen}{RGB}{93,158,164}
\definecolor{purp}{RGB}{128,71,119}
\definecolor{softblue}{RGB}{66,123,161}
\definecolor{darknav}{RGB}{10,17,57}

\definecolor{hotorange}{HTML}{FF5900}
\definecolor{redpurp}{HTML}{3E033D}
\definecolor{purpprup}{HTML}{820082}
\definecolor{dulldarkblue}{HTML}{00003D}

\definecolor{blu}{HTML}{4F8FE3}
\definecolor{gre}{HTML}{208E38}
\definecolor{ora}{HTML}{FF6600}
\definecolor{blutuu}{HTML}{3A4989}

\definecolor{orabro}{HTML}{FF4000}
\definecolor{pinbro}{HTML}{BF0449}
\definecolor{brotwo}{HTML}{7E1E00}
\definecolor{bro}{HTML}{400000}

\definecolor{greagain}{HTML}{42594A}
\definecolor{redagain}{HTML}{D91A1A}
\definecolor{yel}{HTML}{F2BF27}
\definecolor{blugredark}{HTML}{233C50}

\definecolor{cyanish}{HTML}{47D2FF}
\definecolor{midgrey}{rgb}{0.220,0.220,0.251}
\definecolor{redas}{HTML}{F11A22}
\definecolor{darkblu}{HTML}{2B115A}

%\definecolor{}{HTML}{}
%\definecolor{}{HTML}{}
%\definecolor{}{HTML}{}
%\definecolor{}{HTML}{}



%%%%%%%%%%%%%%%%%%%% !!! set these to change colours of poster !!! %%%%%%%%%%%%%%%%%

%% background colour, horizontal lines around title
\newcommand{\csone}{blu}

%% author text colour, \Head (text heading) title colour, box type 1 outline colour
\newcommand{\cstwo}{orabro}

%% box type 2 outline colour, \Subhead (sub heading) text colour
\newcommand{\csthr}{redmaroon}

%% poster title colour, text colour in boxes
\newcommand{\csfou}{black!90}         




%%%%%%%%%%%%%%%%%%%% !!! change these too as req !!! %%%%%%%%%%%%%%%%%



% \newcommand{\titletxt}{Title text that is likely to go over two lines but re-position as required if only one line}
% \newcommand{\personone}{x}
% \newcommand{\persontwo}{y}






%%%%%%%%%%%%%%%%%%%%%%%%%%%%% colour scheme commands %%%%%%%%%%%%%%%%%%%%%%%%%%%%



% \pagecolor{\csone!10}
\newcommand{\textcsone}[1]{\textcolor{\csone}{#1}}
\newcommand{\textcstwo}[1]{\textcolor{\cstwo}{#1}}
\newcommand{\textcsthr}[1]{\textcolor{\csthr}{#1}}
\newcommand{\textcsfou}[1]{\textcolor{\csfou}{#1}}




%%%%%%%%%%%%%%%%%%%%%%%%%% insert user commands here %%%%%%%%%%%%%%%%%%%


\newcommand{\bcen}{\begin{center}}
\newcommand{\ecen}{\end{center}}
\newcommand{\bfr}{\begin{flushright}}
\newcommand{\efr}{\end{flushright}}
\newcommand{\bfl}{\begin{flushleft}}
\newcommand{\efl}{\end{flushleft}}

\newcommand{\benum}{\begin{enumerate}}
\newcommand{\eenum}{\end{enumerate}}
\newcommand{\bit}{\begin{itemize}[label={\color{\csthr}\textbullet}]\itemsep-1pt}
\newcommand{\eit}{\end{itemize}}

\newcommand{\spacebull}{\textcsthr{\textbullet}}

\newcommand{\Head}[1]{\noindent{\large\textcstwo{#1}}\bigskip}
\newcommand{\Subhead}[1]{\noindent{\textcsthr{#1}}}
\newcommand{\inserttitle}[1]{\textcsfou{\textbf{{\veryHuge #1}}}}



%%%%%%%%%%%%%%%%%%%%%%%%%% boxes and such %%%%%%%%%%%%%%%%%%%



\tikzstyle{boxonetemplate} = [draw=\csone, fill=\csone!15, ultra thick,
	rectangle, rounded corners, inner sep=10pt, inner ysep=10pt]
    
\tikzstyle{boxtwotemplate} = [draw=\cstwo, fill=\cstwo!10, ultra thick,
rectangle, rounded corners, inner sep=10pt, inner ysep=10pt]

\tikzstyle{boxthrtemplate} = [draw=\csthr, fill=\csthr!10, ultra thick,
	rectangle, rounded corners, inner sep=10pt, inner ysep=10pt]

\newcommand{\boxone}[3]{%
\vskip3pt%
\begin{tikzpicture}\node[boxonetemplate](box){%
\begin{minipage}{#3\textwidth}%
\bfr\large{\textbf{\textsf{\textcsfou{#1}}}}\efr\vspace{-20pt}
\color{\csthr}{\rule{\textwidth}{0.5mm}}
\textcsfou{#2}
\end{minipage}%
};\end{tikzpicture}%
\vskip3pt%
}


\newcommand{\boxtwo}[3]{%
\vskip3pt%
\begin{tikzpicture}%
\node[boxtwotemplate](box){%
\begin{minipage}{#3\textwidth}%
\bfr\textbf{\textcsfou{#1}}\efr\vspace{-12pt}
\textcsfou{#2}
\end{minipage}%
};\end{tikzpicture}%
\vskip3pt%
}

\newcommand{\boxonent}[2]{%
\vskip3pt%
\begin{tikzpicture}%
\node[boxonetemplate](box){%
\begin{minipage}{#2\textwidth}%
% \bfr\textbf{\textcsfou{#1}}\efr\vspace{-12pt}
\small\textcsfou{#1}
\end{minipage}%
};\end{tikzpicture}%
\vskip3pt%
}

\newcommand{\boxtwont}[2]{%
\vskip3pt%
\begin{tikzpicture}%
\node[boxtwotemplate](box){%
\begin{minipage}{#2\textwidth}%
% \bfr\textbf{\textcsfou{#1}}\efr\vspace{-12pt}
\small\textcsfou{#1}
\end{minipage}%
};\end{tikzpicture}%
\vskip3pt%
}

\newcommand{\boxthrnt}[2]{%
\vskip3pt%
\begin{tikzpicture}%
\node[boxthrtemplate](box){%
\begin{minipage}{#2\textwidth}%
% \bfr\textbf{\textcsfou{#1}}\efr\vspace{-12pt}
\small\textcsfou{#1}
\end{minipage}%
};\end{tikzpicture}%
\vskip3pt%
}

\newcommand{\quickboxone}[4]{%
\begin{textblock}{#3}(#1,#2)%
\boxonent{#4}{#3}%
\end{textblock}
}

\newcommand{\quickboxtwo}[4]{%
\begin{textblock}{#3}(#1,#2)%
\boxtwont{#4}{#3}%
\end{textblock}
}


\newcommand{\quickboxthr}[4]{%
\begin{textblock}{#3}(#1,#2)%
\boxthrnt{#4}{#3}%
\end{textblock}
}




%%%%%%%%%%%%%%%%%%%%%%%%%% arrows and lines %%%%%%%%%%%%%%%%%%%


% requires tikz >= 3.0.0
% also the previous declaration: \usetikzlibrary{arrows,arrows.meta}
% \tikzset{
%   double arrow/.style args={#1}{
%     % -triangle 60,
%     line width=2,\csone % first arrow
%     % postaction={draw,-triangle 60,\cstwo!50,line width=1/4,
%     % shorten <=(#1)/1.2,
%     % shorten >=2*(#1)/1.2} % second arrow
%   }
% }


\newcommand{\downarrowalg}[3]{%
\begin{textblock}{1}(#1,#2)
\begin{center}
\begin{tikzpicture}
\tikzstyle{arrow} = [thick,-,#3!80]
\draw[arrow]
		(0,0) coordinate -- (0,2) coordinate
    node[pos=0.5,inner sep=0pt]{\textcolor{#3!80}{{\footnotesize $\vee$}}};
% \node[right] at (0.8,0.8) {\textcsthr{#3}};
\end{tikzpicture}
% \vspace{-20pt}
\end{center}
\end{textblock}
}


\newcommand{\uparrowalg}[3]{%
\begin{textblock}{1}(#1,#2)
\begin{center}
\begin{tikzpicture}
\tikzstyle{arrow} = [thick,-,#3!80]
\draw[arrow]
		(0,0) coordinate -- (0,2) coordinate
    node[pos=0.5,inner sep=0pt]{\textcolor{#3!80}{{\footnotesize $\wedge$}}};
% \node[right] at (0.8,0.8) {\textcsthr{#3}};
\end{tikzpicture}
% \vspace{-20pt}
\end{center}
\end{textblock}
}

\newcommand{\vlinealg}[3]{%
\begin{textblock}{1}(#1,#2)
\begin{center}
\begin{tikzpicture}
\tikzstyle{arrow} = [thick,-,#3!80]
\draw[arrow]
		(0,0) coordinate -- (0,2) coordinate;
% \node[right] at (0.8,0.8) {\textcsthr{#3}};
\end{tikzpicture}
% \vspace{-20pt}
\end{center}
\end{textblock}
}

\newcommand{\rightarrowalg}[3]{%
\begin{textblock}{1}(#1,#2)
\begin{center}
\begin{tikzpicture}
\tikzstyle{arrow} = [thick,-,#3!80]
% \draw[double arrow=5pt] 
\draw[arrow]
		(5,0) coordinate -- (0,0) coordinate
    node[pos=0.5]{\textcolor{#3!80}{{\footnotesize $>$}}};
% \node[right] at (0.8,0.8) {\textcsthr{#3}};
\end{tikzpicture}
% \vspace{-20pt}
\end{center}
\end{textblock}
}

\newcommand{\leftarrowalg}[3]{%
\begin{textblock}{1}(#1,#2)
\begin{center}
\begin{tikzpicture}
\tikzstyle{arrow} = [thick,-,#3!80]
% \draw[double arrow=5pt] 
\draw[arrow]
		(0,0) coordinate -- (5,0) coordinate
    node[pos=0.5]{\textcolor{#3!80}{{\footnotesize $<$}}};
% \node[right] at (0.8,0.8) {\textcsthr{#3}};
\end{tikzpicture}
% \vspace{-20pt}
\end{center}
\end{textblock}
}

\newcommand{\hlinealg}[3]{%
\begin{textblock}{1}(#1,#2)
\begin{center}
\begin{tikzpicture}
\tikzstyle{arrow} = [thick,-,#3!80]
% \draw[double arrow=5pt] 
\draw[arrow]
		(0,0) coordinate -- (2,0) coordinate;
% \node[right] at (0.8,0.8) {\textcsthr{#3}};
\end{tikzpicture}
% \vspace{-20pt}
\end{center}
\end{textblock}
}


%%%%%%%%%%%%%%%%%%%%%%%%%% start poster %%%%%%%%%%%%%%%%%%%


\begin{document}


%%%%%%%%%%%%%%%%%%%%%%%%%%%%%%%%%%%%%%%%%%%%%%%%%%%%%%%%%%%%%%%%%%%%%%%%%%%%%%%%%%%%%
%%%%%%%%%%%%%%%%%%%%%%%%%%%%%%%%% !!!NOTE!!! %%%%%%%%%%%%%%%%%%%%%%%%%%%%%%%%%%%%%%%%
%%%%%%%%%%%%%%%%%%%%%%%%%%%%%%%%%%%%%%%%%%%%%%%%%%%%%%%%%%%%%%%%%%%%%%%%%%%%%%%%%%%%%
%
% this is how textblock declarations work:
%
% \begin{textblock}{<boxwidth>}(<columnpos>,<rowpos>)
%
% where:
% <boxwidth> 
%    normally is one (i.e. one column, unless you want a 2 column wide block)
% <columnpos> 
%    0,1,2 maps to columns 1,2,3 (the x-co-ord of the textblock's top-left corner)
% <rowpos> 
%    a number 0-100 corresponding to the y-co-ord of the textblock's top-left corner
%%%%%%%%%%%%%%%%%%%%%%%%%%%%%%%%%%%%%%%%%%%%%%%%%%%%%%%%%%%%%%%%%%%%%%%%%%%%%%%%%%%%%
%%%%%%%%%%%%%%%%%%%%%%%%%%%%%%%%%%%%%%%%%%%%%%%%%%%%%%%%%%%%%%%%%%%%%%%%%%%%%%%%%%%%%

% \downarrowalg{0}{1}{\cstwo}
% \rightarrowalg{0}{1}{\cstwo}






%%%%%%%%%%%%%%%%%%%%%%%%%% title/header band %%%%%%%%%%%%%%%%%%%



\begin{textblock}{4}(0,-2.5)
\color{\cstwo}{\rule{1.0\textwidth}{2mm}}
\end{textblock}


\begin{textblock}{4}(0,-1)
\bfl
{\Huge \color{\cstwo}{\textbf{Model building}}}
\efl
\end{textblock}

\begin{textblock}{4}(0,3)
 \color{\cstwo}{\rule{1.0\textwidth}{2mm}}
\end{textblock}




\begin{textblock}{1}(0,46)
\color{\csone}{\rule{1.0\textwidth}{2mm}}
\end{textblock}

\begin{textblock}{1}(0,47.5)
\bfl
{\Huge \color{\csone}{\textbf{User side}}}
\efl
\end{textblock}
    

\begin{textblock}{1}(0,51.5)
  \color{\csone}{\rule{1.0\textwidth}{2mm}}
\end{textblock}





\begin{textblock}{3}(1,46)
  \color{\csthr}{\rule{1.0\textwidth}{2mm}}
\end{textblock}
  

\begin{textblock}{3}(1,47.5)
  \bfr
  {\Huge \color{\csthr}{\textbf{Personalised optimisation}}}
  \efr
\end{textblock}


\begin{textblock}{3}(1,51.5)
   \color{\csthr}{\rule{1.0\textwidth}{2mm}}
\end{textblock}
  


%%%%%%%%%%%%%%%%%%%%%%% text block %%%%%%%%%%%%%%%%%%%
% \begin{textblock}{1}(0,15)

% Words.

% \end{textblock}


%%%%%%%%%%%%%%%%%%%%%%% text block %%%%%%%%%%%%%%%%%%%
% \begin{textblock}{1}(0,20)

%   \boxone{Text title in block style 1}{words in block style 1}{1}
  
% \end{textblock}

% \quickboxtwo{0}{20}{1.3}{%
% xxxxxxxx
% }



%%%%%%%%%%%%%%%%%%%%%%% text block %%%%%%%%%%%%%%%%%%%





\rightarrowalg{0.55}{6}{\cstwo}
\rightarrowalg{1.55}{6}{\cstwo}
\downarrowalg{0}{15}{\cstwo}
\rightarrowalg{0.55}{18}{\cstwo}
\rightarrowalg{1.55}{15}{\cstwo}
\downarrowalg{2.5}{9}{\cstwo}
\downarrowalg{2}{23}{\cstwo}
\downarrowalg{0}{31}{\cstwo}
\rightarrowalg{0.60}{35}{\cstwo}


\rightarrowalg{2.7}{26}{\cstwo}
\vlinealg{2.95}{26.7}{\cstwo}
\leftarrowalg{2.7}{41}{\cstwo}
\vlinealg{2.95}{38.23}{\cstwo}



\quickboxtwo{0}{5}{1}{%
Time-use variables ($D$ modifiable predictors)
}

\quickboxtwo{1}{5}{1}{%
{Transform the time-use data into ilr coordinates \\[-7mm]$$\boldsymbol{z} = V^T \ln \left(\boldsymbol{x}  \right)$$\\[-20mm]}
}


%%%%%%%%%%%%%%%%%%%%%%% arrow %%%%%%%%%%%%%%%%%%%
% \rightarrowalg{0.6}{12}{2}{{\phantom{x}}}
% \rightarrowalg{0}{21}{1}{ilr transform}

\quickboxtwo{2}{5}{1.4}{%
{Multivariate scale ("whiten") $\textit{ilr}$ coordinates $\boldsymbol{z}$ using\\[-4mm]
$$\boldsymbol{z}^{(s)} = \left( \boldsymbol{z} - \hat{\boldsymbol{\mu}_z} \right)^T \hat{\Sigma}_z^{-1/2}$$\\[-20mm]}
}



\quickboxtwo{0}{12}{1}{%
Other covariates (other participant predictors)
}

\quickboxtwo{0}{18}{1}{%
Re-parameterise categorical variables as contrast indicators/"one-hot-coded" w.r.t. to a reference level
}

\quickboxtwo{1}{13.5}{1}{%
{Multivariate scale each covariate group-wise to create block-wise, main-effect design matricies\\[-4mm]
$$\boldsymbol{w}^{(s)} = \left( \boldsymbol{w} - \hat{\boldsymbol{\mu}}_{w} \right)^T \hat{\Sigma}_{w}^{-1/2}$$\\[-20mm]}
}


\quickboxtwo{2.0}{12.5}{1.4}{%
Create all two way interactions as well as squared polynomial terms (for continuous variables only) for the set of multidimensional scaled ilr-coordinates and other group/block-wise scaled predictor variables. Resulting in the group-wise set of predictor matricies
$X^{(g)}$,  $g=1,2,...,G$; where $G$ is the total number of variable groups (interaction/polynomial blocks are seperate predictor groups to main effects)
}



\quickboxtwo{0}{26}{0.97}{
  Cognative variables (outcomes)\\[-7mm]
  $$\boldsymbol{y}_1, \hdots, \boldsymbol{y}_5$$\\[-20mm]
}

\quickboxtwo{0}{34}{0.97}{
  Univariate scale cognative response variables \\[-7mm]
  $$\boldsymbol{y}^{(s)}_1, \hdots, \boldsymbol{y}^{(s)}_5$$\\[-20mm]
}


\quickboxtwo{1.2}{26}{1.4}{%
Fit Group Lasso model for each scaled cognitive outcome (response variable) $\boldsymbol{y}_{r}^{(s)}, \;\; r=1,2,\hdots,5$:
$$
\boldsymbol{\beta}_r^{\ast} = 
\operatorname*{argmin}_{\boldsymbol{\beta}_r}
\bigg| \bigg| 
\boldsymbol{y}_{r}^{(s)} - \sum_{g=1}^{G} X^{(g)} \boldsymbol{\beta}_r^{(g)} 
\bigg| \bigg|_2^{2} + 
\lambda_r \sum_{g=1}^{G} 
\sqrt{{\nu}_g} 
\big| \big| \boldsymbol{\beta}_r^{(g)} \big| \big|_2 
$$
That is, find the regularised/penalised coefficients $\boldsymbol{\beta}_r^{\ast}$ for each response variable block-wise over the scaled main, interaction and polynomial (order 2, continuous only) predictors
}


\quickboxtwo{3}{28}{0.85}{%
Minimise 10-fold cross-validation prediction error over grid of group lasso penaltys to find optimised $\lambda_r^{\ast}$ for each response variable $r$
}






\downarrowalg{0}{60.3}{\csone}
\vlinealg{0}{66}{\csone}
\vlinealg{0}{69.5}{\csone}
\vlinealg{0}{73}{\csone}
\vlinealg{0}{76.5}{\csone}
\hlinealg{0.0985}{80.09}{\csone}
\rightarrowalg{0.45}{79.355}{\csone}
\rightarrowalg{0.51}{64.3}{\csone}






\quickboxone{0}{57}{0.95}{%
Specify response $r$ of interest
}


% \quickboxone{0}{60}{0.95}{%
% Enter current time-use \\[-4mm]
% $$\boldsymbol{x}_{(c)}$$\\[-20mm]
% }



\quickboxone{0}{64}{0.95}{%
Enter demographic info
}





\uparrowalg{1}{61}{\csthr}
\rightarrowalg{1.54}{58.3}{\csthr}
\downarrowalg{2.5}{66}{\csthr}
\rightarrowalg{2.05}{72}{\csthr}
\downarrowalg{2.5}{77}{\csthr}
\downarrowalg{3}{83}{\csthr}
\leftarrowalg{2.54}{89}{\csthr}
\leftarrowalg{1.54}{89}{\csthr}
\leftarrowalg{0.5}{89}{\csthr}





\quickboxthr{1}{64}{0.9}{%
Person specific strata $k$ can be deduced
% }
% \quickboxthr{4.15}{27}{0.9}{%
from the $2^3$=$8$ strata defined by tuples of dichotimised categories (age${}_{<75}$, sex, BMI${}_{<30}$)
}





\quickboxthr{1}{54}{1}{%
Calculate the sample ($\textit{ilr}$-transformed) mean vector and covariance matrix of the $k^{\text{th}}$ strata, $\hat{\boldsymbol{\mu}}_{k}$ and $\hat{\Sigma}_{k}$, respectively 
}


\quickboxthr{2}{54}{1.42}{%
Estimate the surface ("fence") of an ellipsoid (3D generalisation of an ellipse in $D - 1 = 3$-dimensional Euclidean $\textit{ilr}$-space) that captures the 80\% highest density of the $\textit{ilr}$-transformed time-use values specific to the strata
\bit
\item
The fence is calculated assuming a 3-variate Gaussian distribution with points $\boldsymbol{z}^{*}$ that satisfy:
{\\[-4mm]$$
\left( \boldsymbol{z}^{*} - \hat{\boldsymbol{\mu}}_{k} \right)^T \hat{\Sigma}_{k}^{-1} \left( \boldsymbol{z}^{*} - \hat{\boldsymbol{\mu}}_k  \right)
\leq 4.641628
$$\\[-20mm]}
\eit
}








\quickboxthr{1.77}{69}{0.83}{%
Generate grid of 5-min spaced time-use compositions within strata $k$'s contrained space
}

\quickboxthr{2.5}{69}{1.215}{%
{Create exhaustive 5-min spaced grid with points $\boldsymbol{x}_{(\#)}$: ilr transform and only keep those within the ellipsoid %\\[-10mm]
%$$
$\boldsymbol{z}_{(\#)} = V^T \ln \left(\boldsymbol{x}_{(\#)} \right)$
%$$\\[-10mm] 
and then multivariate scale as before to create $\boldsymbol{z}_{(\#)}^{(s)}$
% \\[-4mm]
% $$\boldsymbol{z}_{(\#)}^{(s)} = \left( \boldsymbol{z}_{(\#)} - \hat{\boldsymbol{\mu}}_{z} \right)^T \hat{\Sigma}_{z}^{-1/2}$$
% \\[-20mm]
}
}

\quickboxthr{1}{80}{1.75}{%
Create all scaled main effects, two way interactions as well as squared polynomial terms like before using the personalised user predictors as covariates with the ilr grid points
}

\quickboxthr{3}{87}{0.95}{%
Make predictions for response $r$ from Lasso model ($\boldsymbol{\beta}_r^{\ast}$) over all constrained grid points
}

\quickboxthr{2}{87}{1.0}{%
Extract top 5\% of grid predctions and average associated ilrs to create the optimal ilr coords, $\boldsymbol{z}_{(opt)}^{(s)}$
}

\quickboxthr{1}{87}{1.0}{%
Reverse transform ilr scaling on the optimal ilr and then inverse ilr tranform back to time-use composition scale, $\boldsymbol{x}_{(opt)}$
}


\quickboxone{0}{87}{0.95}{%
Report associated optimal time-use composition $\boldsymbol{x}_{(opt)}$ to the user
}

% \quickboxone{3.9}{88}{1.02}{%
% {Transform compositions to ilr coordinates \\[-10mm]$$\boldsymbol{z}_{(c)} = V^T \ln \left\{\boldsymbol{x}_{(c)}  \right\}$$\\[-10mm] and then scale using strata specific sample ilr mean and covariance \\[-4mm]$$\boldsymbol{z}_{(c)}^{(s)} = \left( \boldsymbol{z}_{(c)} - \hat{\boldsymbol{\mu}}_{k} \right)^T \hat{\Sigma}_{k}^{-1/2}$$\\[-20mm]}
% }


%%%%%%%%%%%%%%%%%%%%%%% text block %%%%%%%%%%%%%%%%%%%
% \begin{textblock}{1}(0,42)

% \boxtwo{Text title in block style 2}{words in block style 2}{1.5}

% \Head{Words }

% Words. 

% \end{textblock}



% %%%%%%%%%%%%%%%%%%%%%%% text block %%%%%%%%%%%%%%%%%%%
% \begin{textblock}{1}(1,62)

% Words.
% \bit
% \item item one
% \item item two
% \eit

% \end{textblock}





% %%%%%%%%%%%%%%%%%%%%%%% text block %%%%%%%%%%%%%%%%%%%
% \begin{textblock}{1}(1,52)

% \Subhead{Text}

% Words

% \end{textblock}



%%%%%%%%%%%%%%%%%%%%%%% text block %%%%%%%%%%%%%%%%%%%
% \begin{textblock}{1}(1,79)

% \Head{Words}

% Words

% \end{textblock}









%\begin{textblock}{1}(2,10)
%%\bibliographystyle{plain}
%\bibliographystyle{ieeetr}
%\bibliography{bib/bibliography.bib}
%\end{textblock}


\end{document}
